\documentclass[11pt,letterpaper,leqno]{amsart}
\usepackage[latin1]{inputenc}  % Unixin merkist?
\usepackage[T1]{fontenc}       % kirjaimet, joissa aksentteja (skandit)


\usepackage{amsfonts}         
\usepackage{amsmath}
\usepackage{amssymb}
\usepackage{amsthm}
\usepackage{ae,aecompl,amsbsy}
\usepackage{epsf}
\usepackage{graphics}
\usepackage{ucs}
\usepackage[pdftex]{graphicx}
\usepackage{amsaddr}
%\usepackage[foot]{amsaddr}


\usepackage{color}
\usepackage{verbatim}
\usepackage{algorithm}
\usepackage{algorithmic}
\usepackage{caption}
\usepackage{subcaption}

%bibligraphy
\usepackage[comma,sort]{natbib}
\usepackage{hyperref} 

\newcommand{\R}{\mathbb{R}}
\newcommand{\C}{\mathbb{C}}
\newcommand{\Q}{\mathbb{Q}}
\newcommand{\N}{\mathbb{N}}
\newcommand{\Z}{\mathbb{Z}}

\newcommand{\divergence}{\operatorname{div}}
\DeclareMathOperator*\argmin{argmin}

\newcommand{\TODO}[1]{\textcolor{red}{{\sc Todo:} #1}}


\newtheorem{definition}[equation]{Definition}
\newtheorem{proposition}[equation]{Proposition}
\newtheorem{assumption}[equation]{Assumption}
\usepackage{lineno}


\numberwithin{equation}{section}
\graphicspath{ {figures/} }

\title{Analytical results and physical understanding of demo: Uniform Electric field illuminating a sphere in a uniform earth}
%\author{Team C}

                               
\begin{document}

%\linenumbers
\maketitle
\thispagestyle{empty}

\bibliographystyle{plainnat}


\normalsize

\vspace{0.4cm}


Let consider a resistive uniform wholespace enclosing a sphere with conductivity $\sigma_1$ and radius $R$.  The background has conductivity $\sigma_0$. We have uniform, unidirectional, static electric field $E_0$ going through this space.

Related Maxwell's equations: 
\begin{equation}
\begin{aligned}
  \nabla \times E & = 0 \\
  \nabla\times H & = J \\
  J  =&  \sigma E.
\end{aligned}
\end{equation}

$E_0$ induces charge in the sphere which induces  new $E_s$  ?
\TODO{What's really happening? }

\section{Potentials}

Total potential outside the sphere can be expressed as   
\begin{equation}
V_t  = -E_0x (1-\frac{\sigma_1-\sigma_0}{\sigma_1+2\sigma_0} \frac{R^3}{r^3})
\end{equation}
and inside the sphere
\begin{equation}
 V_t= -E_0 x\frac{3 \sigma_0}{\sigma_1+2\sigma_0}.
 \end{equation}
The primary potential is related to the uniform electric field and the secondary is 

\section{Electric Fields}
Since the first Maxwell's equation the electric field equals to the negative gradient of the potential:
\begin{equation}\label{eq:Epotential}
E = -\nabla V
\end{equation}

\TODO{Why electric field only is in only x direction inside the sphere? Why there is discontinuity?}

The general solution of Laplace's equation in spherical coordinate is
\begin{equation}\label{eq:gsln}
V = \sum\limits_{n=0}^\infty\sum\limits_{m=0}^n (A_{nm}r^n + \frac{B_{nm}}{r^{n+1}})P_n^m(\cos\theta)e^{im\phi}
\end{equation}  
Assuming the primary uniform electric field $E_0$ is restricted to x direction, the primary potential by integral of eq.\eqref{eq:Epotential} is
\begin{equation}\label{eq:primary}
V_p = -E_0 x = -E_0 r\cos\theta = -E_0 r P_1(\cos\theta)
\end{equation}
Comparing eq.\eqref{eq:gsln}\eqref{eq:primary} and considering boundary conditions, we observe that there must be $m=0$ and $n=1$ in the secondary potential expression. So the secondary potentials reduces to
\begin{equation}
V_s = [A_1r+Br^{-2}]P_1(\cos\theta)
\end{equation}
Considering the boundary conditions $V|_{r=\infty}=0$ and $V|_{r=0}=0$, the secondary potential outside the sphere ($r>R$) is
\begin{equation}
V_1^s = B_1r^{-2}P_1(\cos\theta)
\end{equation}
and inside the sphere ($r<R$) is
\begin{equation}
V_2^s = A_1rP_1(\cos\theta)
\end{equation}
So the total potential according to superposition theorem is then
\begin{align}
V_1 &= [-E_0r + B_1r^{-2}]P_1(\cos\theta) \\
V_2 &= A_1rP_1(\cos\theta)
\end{align}
On the surface of the sphere, consider that both the normal current density and the potential must be continuous across the surface of the sphere, we have
\begin{align}
-\sigma_1A_1 &= 2\sigma_0B_1R^{-3} + \sigma_0E_0 \\
A_1R &= -E_0R + B_1R^{-2}
\end{align}
Solving above equations gives us the total potential
\begin{align}
V_1 &= -E_0r\cos\theta(1 - \frac{\sigma_1-\sigma_0}{\sigma_1+2\sigma_0}\frac{R^3}{r^3}), \quad r > R \\
V_2 &= -\frac{3\sigma_0}{\sigma_1+2\sigma_0}E_0r\cos\theta \quad r\leq R
\end{align}
Note that $x = r\cos\theta$. According to eq.\eqref{eq:Epotential}, the electric field has only x component inside the sphere.






\section{Current Densities}
For  current densities and electric field there is the following relation:  
\begin{equation}
J = \sigma E.
\end{equation}
The total current density is (always?)  continuous, but as can be seen from the pictures of our demo, the primary and the secondary current densities can be highly discontinuous.

\section{Charge Accumulation}

Gauss's law of Maxwell's equation says that 
\begin{equation}
\nabla\cdot E = \frac{\rho}{\epsilon_0},
\end{equation}
where $\rho$ is the total electric charge density. Charge is build on the surface of the sphere.

% and where we have 
%\begin{equation}
%\rho = E_0 \frac{\epsilon_0 3 x}{R^3} \frac{\sigma_1-\sigma_0}{\sigma_1+2\sigma_0}.
%\end{equation}


%See S. Ward and G. Hohmann for more details.

See EMGroupArchiveJuly2012.pdf and Ward and Hohmann for more details.
\bibliography{refs}
\end{document}